\PassOptionsToPackage{usenames,dvipsnames}{xcolor}
\documentclass[twocolumn,twocolappendix]{aastex63}

% Add your own macros here:
\pdfoutput=1 %for arXiv submission
%\usepackage{amsmath,amssymb,amstext}
\usepackage{amsmath,amstext}
\usepackage[T1]{fontenc}
\usepackage{apjfonts}
\usepackage{ae,aecompl}
\usepackage[utf8]{inputenc}
\usepackage[figure,figure*]{hypcap}

\usepackage{url}
\urlstyle{same}

\usepackage{lineno}
\linenumbers
%\modulolinenumbers[2]

\newcommand{\placeholder}[1]{\textit{PLACEHOLDER: #1}}


% header settings
\shorttitle{Differentiable Lensing Simulations}
\shortauthors{Lanzieri et al.\ (LSST~DESC)}

% ======================================================================

\begin{document}
\title{Forecasting the power of Higher Order Weak Lensing Statistics with automatically differentiable simulations}
\collaboration{1000}{The LSST Dark Energy Science Collaboration (LSST DESC)}
\noaffiliation

% paper leads
\author{Denise Lanzieri}
\affiliation{Université Paris Cité, Université Paris-Saclay, CEA, CNRS, AIM, F-91191, Gif-sur-Yvette, France}

\author[0000-0001-7956-0542]{Fran\c{c}ois Lanusse}
\affiliation{Université Paris-Saclay, Université Paris Cité, CEA, CNRS, AIM, 91191, Gif-sur-Yvette, France; }

%First Tier authors (alphabetical)
\author{Chirag Modi}
\affiliation{Center for Computational Astrophysics,
Center for Computational Mathematics,
Flatiron Institute, New York, NY 10010, USA}
% \noaffiliation

\author{Benjamin Horowitz}
\affiliation{Lawrence Berkeley National Laboratory, 1 Cyclotron Road, Berkeley, 94720, CA, USA}
\affiliation{Department of Astrophysical Sciences, Princeton University, Princeton, NJ 08544, USA}


\author{Joachim Harnois-Déraps}
\affiliation{School of Mathematics, Statistics and Physics, Newcastle University, Herschel Building, NE1 7RU Newcastle-upon-Tyne, UK}
% Second Tier authors (alphabetical)
\author{Jean-Luc Starck}
\affiliation{Université Paris-Saclay, Université Paris Cité, CEA, CNRS, AIM, 91191, Gif-sur-Yvette, France; }

\begin{abstract}
In this project we investigate the relative constraining power of various map-based higher order weak lensing statistics in an LSST setting. Such comparisons are typically very costly as they require a large number of simulations, and become intractable for more than 3 or 4 cosmological parameters. Instead we propose a new methodology based on the TensorFlow framework for automatic differentiation, and more specifically the public FlowPM N-body simulation code. By implementing ray-tracing in this framework, we will be able to simulate lensing lightcones, and compute Higher Order lensing statistics on the resulting maps. These statistics can then be differentiated with respect to cosmology, or any systematics included in the simulations, thus allowing, for exact Fisher forecasts.
\end{abstract}

\keywords{methods: statistical -- dark energy}

%\accepted{}
%\submitjournal{the Astrophysical Journal Supplement}


%\tableofcontents%-----------------------------
%===========================
% BEGINNING OF THE MAIN TEXT
%===========================

\section{Introduction}

\section{FlowPM}

\section{Ray-tracing}
In this section, we provide a brief overview of the lensing convergence and the basic concepts of cosmic shear.
Weak gravitational lensing is a promising observational technique to infer the projected matter density distribution between an observer and a source by measuring galaxy shape correlations. 

In a homogeneous FLRW Universe, the geodesic equation of Light ray trajectories induced by the density fluctuations are expressed as :
\begin{equation}\label{geod}
    \frac{d^2\textbf{x}_{\bot}(\chi)}{d\chi^2}
    = -2 \nabla_{\bot} \Phi(\textbf{x}_{\bot},\chi).
\end{equation}
If we integrate Eq.\ref{geod} over the line of sight along $\chi'$ the solution $\textbf{x}_{\bot}$ for the radial comoving distance becomes:
\begin{equation}
    \textbf{x}(\chi)=f_k(\chi)\boldsymbol{\theta}-\frac{2}{c^2} 
    \int_0^{\chi} d\chi'f_k(\chi-\chi')[\boldsymbol{\nabla}_{\bot}\Phi(\textbf{x},\chi')-\boldsymbol{\nabla}_{\bot}\Phi^{(0)}(\chi')]
\end{equation}
 where we indicate as $f_k(\chi)$ and $\chi$ the angular and radial comoving distance, respectively.
 In the absence of lensing the observer would see the distance \textbf{x} under an angle $\boldsymbol{\beta}$$=\textbf{x}(\chi)/f_k(\chi)$.
 
 In the presence of lensing the separation vector \textbf{x} will be seen under an angle $\boldsymbol{\alpha}$, defining as:
 \begin{equation}
     \boldsymbol{\alpha}=\frac{2}{c^2} \int_0^{\chi} d\chi'
     \frac{f_k(\chi-\chi')}{f_k(\chi)}
     [\boldsymbol{\nabla}_{\bot}\Phi(\textbf{x},\chi')-\boldsymbol{\nabla}_{\bot}\Phi^{(0)}(\chi')].
 \end{equation}
 So that, the difference between the apparent angle $\theta$ and the source position $\beta$ defines the lens equation:
 \begin{equation}
 \boldsymbol{\alpha=\theta - \beta}.
 \end{equation}
In the next section we will infer the lensing potential  $\Phi$ by solving the Poisson equation from the three dimensional density contrast $\delta$:
\begin{equation}\label{Poisson}
    \nabla_{\bot}^2 \Phi(x_{\bot},\chi)=
    \frac{3H_0^2 \Omega_m}{2c^2a(\chi)}
    \delta(x_{\bot},\chi)
\end{equation}
where we indicate as $a$ the scale factor, and  as $\Omega_m$ and $H_0$ the matter density and the Hubble parameter, respectively. 

\subsection{The Born approximation}


In this work we implement the Born approximation, integrating the potential gradient along the unperturbed ray. If we consider the series expansion in powers of $\Phi$, assuming that the $\Phi$ is small and considering just the first-order approximation, the gravitational potential  can be written as:
\begin{equation}
    \Phi(f_k(\chi')\boldsymbol{\theta},\chi').
\end{equation}
We can now define the Jacobian matrix \textbf{A}$=\partial\boldsymbol{\beta}/\partial\boldsymbol{\theta}$, describing the linear mapping from the lensed image to the unlensed source:
\begin{align}
    A_{ij}=& \frac{\partial\beta_i}{\partial\theta_j}=
    \delta_{ij}-\frac{\partial\alpha_i}{\partial\theta_j} \\
     & =\delta_{ij}-\frac{2}{c^2}
    \int_0^{\chi} d\chi'
     \frac{f_k(\chi-\chi')f_k(\chi')}{f_k(\chi)}
     \frac{\partial^2}{\partial x_i \partial x_j} \Phi(f_k(\chi')\boldsymbol{\theta},\chi').
\end{align}
From the parametrization of the symmetrical matrix \textbf{A}, we can define the spin-two shear $\gamma=(\gamma_1,\gamma_2)$ and the scalar convergence field, $\kappa$. 
Considering this definition the convergence and the shear are defined as second derivative of the potential:
\begin{equation}\label{kshear}
    \kappa=\frac{1}{2}(\partial_1\partial_1+\partial_2\partial_2)\psi; \ \gamma_1=\frac{1}{2}(\partial_1\partial_1-\partial_2\partial_2)\psi; \ 
    \gamma_2=\partial_1\partial_2\psi;
\end{equation}
where we define the 2D \textit{lensing potential} $\psi$,
\begin{equation}
    \psi(\boldsymbol{\theta},\chi)=
    -\frac{2}{c^2}
    \int_0^{\chi} d\chi'
     \frac{f_k(\chi-\chi')f_k(\chi')}{f_k(\chi)}
    \Phi(f_k(\chi')\boldsymbol{\theta},\chi').
\end{equation}
The two fields $\gamma$ and $\kappa$ describe the distortion in the shape of the image, and the change in the angular size, respectively.
Considering the Eq.\ref{Poisson} and the Eq.\ref{kshear} the convergence $\kappa$ becomes:
\begin{equation}
    \kappa_{born}(\boldsymbol{\theta},\chi_s)= \frac{3H_0^2 \Omega_m}{2c^2}
    \int_0^{\chi_s} 
    \frac{d\chi'}{a(\chi')}
     \frac{f_k(\chi-\chi')}{f_k(\chi)}
     f_k(\chi')
    \delta(f_k(\chi')\boldsymbol{\theta},\chi').
\end{equation}

%\begin{equation}
%where $n(\chi)$ is a redshift independent factor of normalization from Poisson equation.
 %   n(\chi)=(AB)(\chi^2)
%\end{equation}
%where $A$ is the two dimensional mesh area in $rad^2$ per pixel and $B$ is the mean 3 dimensional particle density in the all cube.


\section{Methods}

\section{Results}

\section{Discussion}

\subsection{Future directions}

\section{Conclusions}

%=====================
% END OF THE MAIN TEXT
%=====================

\end{document}
